\documentclass[11pt]{article}
\usepackage[margin=0.6in]{geometry}
\usepackage{booktabs}
\usepackage{multirow}

\title{CS224S Assignment 3}
\author{Milind Ganjoo\\\texttt{mganjoo@cs.stanford.edu}
\and%
Stephanie Lynne Pancoast\\\texttt{pancoast@stanford.edu}
\and%
Sebastian Schuster\\\texttt{sebschu@stanford.edu}}
\date{}

\begin{document}

\maketitle
\section{Improving the monophone acoustic model}

In part 1 we tried various values for the five parameters to determine the
optimal values for the male-only training set with using Mel frequency
cepstral coefficients with no normalization or delta coefficients (as these
are explored in later parts of the assignment). In this section we describe
our method for tuning the parameters, present the results, and include a
discussion as to why certain parameter adjustments showed improved test
results with respect to the baseline system.

To initially gain a sense of the performance with respect to each parameter,
we went through each parameter and adjusted the value while fixing the others.
We also adjusted the run.sh script to print out the training error to give us
insight into whether the model was over- or underfitting the training set. As
expected, we observed dependencies between certain parameters and while moving
forward we would tune these dependent parameters together.

\subsection{Number of iterations}
We tried 10, 20 and 30 iterations for the found 20 iterations to show the
lowest word and sentence error rate. Ten iterations was not sufficient for the
Gaussian HMM model to converge, whereas after thirty iterations the
improvement seen at twenty has plateaued and begun to worsen. This is
confirmed by noting a slight increase in the test error and decrease in train
error for both WER and SER.\@

\subsection{Boost silence likelihoods}
The default value for this parameter was 2.0, indicating no special treatment
being given to the likelihoods of GMMs corresponding to the silence phones. By
increasing this parameter, we are indicating it is more likely for a silence
phone to occur, and a decrease in the parameter makes the model less likely to
choose a silence phone. We tried values between 0.3 and 1.5, and found the
best performance results by setting \texttt{boost\_silence} to 0.5. This is equivalent
to halving the weights for all models that correspond to the silence
decreased.

The silence class is of very different nature than the other phone classes.
When we have a dataset that fits the test set well we can rely on the silence
models to be accurately trained.  When we train the phones well, we  need not
reduce the silence likelihood. However, since we do not have much training
data in this scenario and we know the training data does not fit the test data
well, because we are only training on males and the test set is mixed gender
(as is discussed later), reducing the likelihood helps the system. It
essentially says only choose silence if you are extra certain it is a silence
segment.

\subsection{Total Gaussians}
We experimented with Gaussian sizes between 100 and 612, finding 256 performed
the best. The total number of Gaussians is the largest possible number of
Gaussians in the GMM which models each subphone. Too few Gaussians (e.g 100)
under fits the data, while too large a value (e.g. 612) overfits. Comparing a
system with the optimal parameters and one with all the optimal parameters
except totgauss which is doubled, the training error decreases for both WER by
0.04 and 0.13 while the test error increases by 3.53 and 4.14---confirming
that we are indeed overfitting.

\subsection{Max Iter Inc}
For the male-only configuration, we found that allowing the number of
Gaussians to be increased until the final iteration performed the best. In
this training scenario we have limited data, so when the number of Gaussians
are split they already are a good representation of the data and do not need
more iterations to be re-fit. Whichever number of iterations we chose for the
system, we set \texttt{max\_iter\_inc} to one value less.

\subsection{Realign Iters}
We found that regardless of the other parameter values, the system performs
best when the forced alignment is performed at every iteration. This is not
surprising as it repeatedly re-estimating the HMM parameters ensures the HMMs
are fitting the training data as closely as possible. Whichever number of
iterations we chose for the system, we did forced alignment at every
iteration.

\subsection{Final Adjustments}
Once we determined the best value for a parameter keeping the others fixed, we
adjusted the other parameters while keeping the given one at its best value.
By doing this we found the best performing setup to be:

\begin{itemize}
  \item 20 iterations
  \item Silence boosted by 0.5
  \item Maximum of 256 Gaussians
  \item Gaussians allowed to split until iteration 19
  \item Realignment iterations: $[1, 2, \ldots, 19]$ (after every iteration)
\end{itemize}

This gave us the error rates summarized in Table~\ref{tab:wer-mono}. On the test
set, this model gave 554 insertions, 383 deletions, and 1688 substitutions.

\begin{table}[h]\centering
  \begin{tabular}{lcc}
    \toprule
    & train & test \\
    \midrule
    WER & 0.67 & 9.18 \\
    SER & 2.21 & 18.96 \\
    \bottomrule
  \end{tabular}
  \caption{Error rates for improved monophone acoustic
  model}\label{tab:wer-mono}
\end{table}

\section{Improving feature extraction}

In part 2 we improved feature extraction by adding a normalization step after
the extraction was done to filter out environmental noise. With the male-only
training set this step lowers both the WER and the SER by approximately a
quarter to 6.79\% and 14.86\%, respectively.

While the number of deletions did not change a lot, we saw a significant drop
in the number of insertions and substitutions. This indicates that, on the one
hand, background noise is less frequently mapped to a phone and, on the other
hand, that the model is better at distinguishing between phones as without the
noise it should be easier to map the features to a certain phone.

We also tried some other model parameter combinations but consistently got the
best results with the parameters from Step 1.

By optimizing the model parameters and by including the noise filtering step,
we were able to lower the error rate by almost 50\% over our baseline. However,
a seemingly low WER of 6.79\% still implies that one has to repeat a 16 digit
number sequence approximately 67.5\% of the time which is clearly not
practical. In the next section, we try to improve our system by using
different and more training data.

\section{Different training data}

We first used our optimal parameters from Part 1 and trained new models using
the following training sets:

\begin{itemize}
  \item 10 male speakers
  \item 10 female speakers
  \item 5 male and 5 female speakers
  \item 55 male and 57 female speakers
\end{itemize}

The results are presented in Table~\ref{tab:wer-all}\@. First, we can see that training only
on female speakers seems to work better than training only on male speakers.
There are two possible explanations for that. Either the test set contains
more female speakers than male speakers or the set of female speakers has more
variability than the set of male speakers. In such a small random sample it is
not unlikely that there is for example a greater variety of accents in the
female sample.

Further, we can see that including both male and female speakers decreases the
error rate by more than 70\% compared to only using female speakers. This is not
surprising as we have speakers of both gender in our test set, so the training
data is more similar to the test data.

If we include a lot more training examples as it is the case in training set 4,
we see another small gain over the small mixed-gender set. The larger training
error is probably caused by the larger variety in the training set but the fact
that the training error and the test error are almost the same indicates that
the training data is well suited to train a model that works well on a large
test set.

\begin{table}\centering
  \begin{tabular}{@{}rcrrcrrcrrr@{}}\toprule%
    \multirow{2}{*}{Dataset} & \phantom{a} & \multicolumn{2}{c}{train}
    & \phantom{a} & \multicolumn{2}{c}{test} & \phantom{a}
    & \multicolumn{3}{c}{changes}\\
    \cmidrule{3-4} \cmidrule{6-7} \cmidrule{9-11}
    && WER & SER && WER & SER && Ins & Del & Sub\\ \midrule%
    male && 1.55 & 1.82 && 6.79 & 14.86 && 367 & 376 & 1198\\
    female && 0.76 & 2.11 && 5.35 & 12.34 && 204 & 420 & 905\\
    mixed && 0.55 & 1.82 && 1.51 & 4.57 && 116 & 200 & 115\\
    full && 1.04 & 3.11 && 1.47 & 4.23 && 138 & 128 & 154\\
    \bottomrule
  \end{tabular}
  \caption{Error rates for all datasets}\label{tab:wer-all}
\end{table}

We also did some parameter tuning for test sets 2 through 4. The optimal
parameters and the corresponding test WER and SER are presented in
Table~\ref{tab:wer-param}.

\begin{table}\centering
  \begin{tabular}{@{}rcrrrrrcrr@{}}\toprule%
    Dataset & \phantom{a} & \texttt{num\_iterations} & \texttt{boost\_silence}
    & \texttt{totgauss} & \texttt{realign\_iters} & \texttt{max\_iter\_inc}
    & \phantom{a} & WER & SER\\ \midrule%
    1 && 20 & 0.5 & 256 & 1:1:19 & 19 && 6.79 & 14.86\\
    2 && 20 & 0.5 & 512 & 1:1:19 & 19 && 5.2 & 11.72\\
    3 && 20 & 0.75 & 256 & 1:1:19 & 19 && 1.7 & 4.95\\
    4 && 20 & 1.5 & 2048 & 1:1:19 & 10 && 0.83 & 2.57\\
  \end{tabular}
  \caption{Error rates with various parameters tuned}\label{tab:wer-param}
\end{table}

These results indicate a few trends. First, the more data we add and the more
representative data we have, the more gaussians we can use to estimate the
model. Also, we got better results in case we set the last iteration at which
the number of gaussians is increased to a lower value for the more general data
sets 3 and 4. This indicates that if the training data is more representative,
then we want to fit the gaussians for more iterations to the training data
instead of generating more and more gaussians that we only fit for fewer
iterations. Lastly, in case we have better data to train the phones on, it seems
to help to increase the \texttt{boost\_silence} parameter.

By using a larger and  more representative data set we were able to lower the
WER to 0.83. At this error rate, a bank customer would have to repeat a 16-digit
number on average in only 12\% of the cases which makes the system a lot more
practical.

\subsection{Adding delta features}

The final enhancement to our model was to enable an additional training step:
the generation of delta features. Since delta features capture changes in
formant and signal nature from frame to frame, and can provide useful cues
towards phone identity, we expected them to provide improved performance.

We tuned two parameters in the delta training step: the number of initial
Gaussians to start with (\texttt{numleaves}) and the number of final Gaussians
(\texttt{totgauss}). In general, we found that a much larger number of final
Gaussians was required for an improvement over the simple monophone model
(values of \texttt{totgauss} smaller than the ones chosen previously obviously
resulted in a worse WER).

We also observed a diminishing returns effect with perfecting the values of the
delta model parameters, in that the marginal improvement obtained with higher
values of parameters was not significant.

Eventually, we found that the best choice for the parameters was:
\begin{itemize}
  \item Maximum number of Gaussians (\texttt{totgauss}): 8000
  \item Initial number of leaves (\texttt{numleaves}): 100
\end{itemize}

With these parameters, we obtained a slight improvement in our system's
performance, as summarized in Table~\ref{tab:wer-delta}.

\begin{table}[h]\centering
  \begin{tabular}{lcc}
    \toprule
    & train & test \\
    \midrule
    WER & 0.17 & 0.54 \\
    SER & 0.55 & 1.7 \\
    \bottomrule
  \end{tabular}
  \caption{Error rates for model with delta features}\label{tab:wer-delta}
\end{table}

\subsection{Benefits of using delta features}
To determine qualitatively where delta features helped, we compared the output
of the monophone-only model, the model including delta features, and the gold
set of decodings in the test set. We made a few interesting observations:

\begin{itemize}
  \item With delta features, the model was able to differentiate ``1'' and ``9''
  better in a few cases. For example, the gold sequence ``7 7 8 9 7 1 9'' was
  mistakenly decoded as ``7 7 8 9 7 1 1'' by the monophone-only model, but was
  correctly decoded by the delta-enhanced model. We surmise that the ability to
  capture formant changes across frames allowed the latter model to observe the
  forceful dental ``release'' at the beginning of ``nine'', even if the vowel
  pronunciation may have sounded similar to ``one''.
  \item The delta-enhanced model was also able to distinguish sequences of
  similar, vowel-only phones better. For example, in a sequence involving two
  consecutive ``oh'' digits (``4 2 8 0 0 9''), the monophone-only model observed
  only one digit, while the delta-enhanced model observed both. Again, this is
  probably due to the improved ability to capture the subtle formant change
  between two utterances of ``oh''.
  \item However, delta features did not always improve recognizability. The
  delta-enhanced model inserted an extraneous ``8'' in the sequence ``3 5 4 4 z
  6 z'', making it ``3 5 4 4 8 z 6 z''. It appears that the same ability to
  better identify forceful releases at the beginning of a phone that allowed the
  model to distinguish ``one'' and ``nine'' caused the model to be overly
  cautious and detect a transition from an ``eight'', which ends in a dental
  sound, to a ``zero'', which begins with a sibilant.
\end{itemize}


\end{document}
